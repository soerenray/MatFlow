\section{Datenbankverbindung}

\subsection{Datenbankaufbau}
Die Datenbank beinhaltet folgende Tabellen:\\
\textnormal{$\underline{Primärschlüssel}$},
\textnormal{$Fremdschlüssel_{Name\:in\:Ursprungstabelle}$}

\begin{itemize}[noitemsep]
	\item Server
	\begin{itemize}[noitemsep]
		\item $\textnormal{\underline{server\_ID}}$ :String
		\item name :String
		\item adress :String
		\item limitWorkflows :int
	\end{itemize}
	\item User
	\begin{itemize}[noitemsep]
		\item $\textnormal{\underline{email}}$ :String
		\item name :String
		\item permission :int
		\item password :String
		\item hashnr :int
	\end{itemize}
	\item Workflow\_Template
	\begin{itemize}[noitemsep]
		\item \textnormal{$\underline{name}$} :String
		\item \textnormal{$creator_{email\:in\:User}$} :String
		\item dag :File
		%muss ich noch recherchieren wie genau files gespeichert werden können
	\end{itemize}
	\item Workflow
	\begin{itemize}[noitemsep]
		\item $\textnormal{\underline{name}}$ :String
		\item dag :File
		\item files :File
	\end{itemize}
	\item Version
	\begin{itemize}[noitemsep]
		\item $\underline{workflow}_{name\:in\:Workflow}$ :String
		\item $\underline{version}$ :String
		\item $creator_{email\:in\:User}$ :String
		\item conf :file
	\end{itemize}
	\item Execution
	\begin{itemize}[noitemsep]
		\item $\underline{workflow}_{name\:in\:Workflow}$ :String
		\item $\underline{version}_{in\:Version}$ :String
		\item $\underline{server\_ID}_{in\:Server}$ :String
		\item $executor_{email\:in\:User}$ :String
		\item start :timestamp
		\item stop :timestamp
		\item result :file
	\end{itemize}
\end{itemize}

\subsection{Database Package}

Hier Package png und Erklärung

\subsubsection{Class.DatabaseTable}
DatabaseTable ist die einzige Klasse die direkt mit der Datenbank kommuniziert. Ihre Funktion ist hauptsächlich MySQL Befehle entgegennimmt, diese an die Datenbank weitergibt und die dadurch erhaltenen Antworten zurückgibt.

\paragraph{set(create: String):String} Erstelle neuen Eintrag. Versuche $create$ auf MySQL auszuführen und gib die Antwort als String zurück.

\paragraph{delete(del: String):String} Lösche einen existierenden Eintrag. Versuche $del$ auf MySQL auszuführen und gib die Antwort als String zurück.

\paragraph{modify(change: String):String} Ändere einen existierenden Eintrag. Versuche $change$ auf MySQL auszuführen und gib die Antwort als String zurück.






+getUser(UID:Int): String
+getUserAll(): String

+getTemplate(tID: Int); String
+getTemplate(): String
+getWorkflow(wID: Int) :String
+getWorkflowAll(): String
+getWorkflowVersion(wID: Int, vID:String):String
+getWorkflowVersionAll(wID: Int): String


+getServer(sID: Int): String
+getServerAll(): String

+getPassword(uID: Int): (Int, String)





















\newpage