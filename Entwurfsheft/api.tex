\subsubsection{API Schnittstelle}
\paragraph{Class: FrontendAPI}
Diese Klasse ist der Dreh- und Angelpunkt der ganzen Anwendung. Alle Anfragen von der Client Applikation werden an diese API, 
entworfen im Bequemlichkeitsmuster der \textbf{Fassade}, (im JSON Format) geschickt. Die Entscheidung FrontendAPI als 
\textbf{Singleton} zu entwerfen besteht darin, dass die API eine global ansprechbare Instanz sein soll, die auf dem Server an 
einem bestimmten Port laufen soll. Wie dies geschieht sind Implementierungsdetails, bei denen das eingebundene Framework 
\textbf{Flask} hilft, muss aber in die Entwurfsentscheidung einbezogen werden.
\subparagraph{Constructor}

\subparagraph{Version(versionNumber:VersionNumber, note:String):Version}
%bei derart trivialen Konstruktoren/Methoden spare ich mir einen Funktionskommentar

\subparagraph{Parameters}
\begin{itemize}
	\item{versionNumber:}
	Number that identifies the new Version
	\item{note:}
	TODO
\end{itemize}