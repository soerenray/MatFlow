\subsubsection{API Package}
\paragraph{Package: api} This package is the interface between the client's application and the server application. 
All requests are sent to this API(package) in JSON format.
\paragraph{Class: FrontendAPI}
This class is the main interface of the whole application.The decision to design FrontendAPI as
 \textbf{Singleton} pattern means that the API should be a globally accessible instance that runs on the server
on a certain port. The way this happens happens is an  implementation detail with which the framework
\textbf{Flask} helps, but the base idea must be respected in the software design decision. The client application
can get the api status code to obtain information regarding the execution of api calls (e.g. status code
changes when an exception has been thrown)

\paragraph{get\texttt{\_}FrontendAPI():FrontendAPI} returns the FrontendAPI in singleton design fashion, meaning there is only one instance
of FrontendAPI in circulation at all times.

\paragraph{get\texttt{\_}status\texttt{\_}code():int} return status code  

\paragraph{set\texttt{\_}status\texttt{\_}code(status\texttt{\_}code:int)}
sets the status code, only performed by the \textbf{ExceptionHandler}
\begin{itemize}
        \item \textbf{status\texttt{\_}code}
        the status code showing the state of the latest API call
\end{itemize}

\paragraph{get\texttt{\_}server\texttt{\_}details():String}
gets all server details (container limit, cpu resources, ..) and returns them in a json format
(json.dumps is interpreted as String in native python)

\paragraph{set\texttt{\_}server\texttt{\_}details(json\texttt{\_}details): String)}
sets all server details (container limit, cpu resources, ..) in a json format
(json objects are interpreted as String in native python). Only one bulk update as a whole
due to long delay with server communication from client's perspective
\begin{itemize}
        \item \textbf{server\texttt{\_}details}
        all server details in json format
\end{itemize}

\paragraph{get\texttt{\_}all\texttt{\_}users\texttt{\_}and\texttt{\_}details(): String}
gets all server details (container limit, cpu resources, ..) in a json format
(json objects are interpreted as String in native python). Only one bulk update as a whole
due to long delay with server communication from client's perspective




