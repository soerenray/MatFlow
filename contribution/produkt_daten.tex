\section{Produkt-Daten}
% unter Produktdaten geht es darum, welche Daten gespeichert und genutzt werden,
% aber nicht wie. Der Datenbankaufbau wird Teil des Entwurfsphase sein.


\newcommand{\todo}{\textcolor{red}{\textbf{ TODO }}}
\newcommand{\discuss}{\textcolor{green}{\textbf{ DISCUSS }}}
\newcommand{\intern}[1]{<\emph{#1}>}
\newcommand{\head}[1]{\vspace{0.2cm}\item[\textbf{#1}]\ \vspace{0.1cm}}
\newcommand{\kriterium}[2]{\vspace{0.2cm}\item[\textbf{#1}] \textbf{\emph{#2} wird umgesetzt durch:\ }\vspace{0.1cm}}

\newenvironment{anforderung}[1]{
	\begin{enumerate}[label=/#1\arabic*0/, leftmargin=1.6cm, noitemsep]}
	{\end{enumerate}
}
\newenvironment{test_env}[1]{
	\begin{enumerate}[label=/#1\arabic*0/, leftmargin=1.6cm]}
	{\end{enumerate}
}
\newenvironment{anforderung_start_at}[2]{\begin{enumerate}[label=/#2\arabic*0/, start={#1}, leftmargin=1.6cm, noitemsep] }{\end{enumerate}}
%\newenvironmentx*{subanforderung}[2][1=1]
%    {\begin{enumerate}[label=/#2\arabic{enumi}\arabic*/, start={#1}]}
%    {\end{enumerate}}





\begin{anforderung}{D}
    \item \label{experiment_daten}\gls{experiment}daten
    \begin{subanforderung}{D}
        \item Name
        \item Beschreibung
        \item Parameter
        \item Status
        \item \glspl{tag}s
        \item Zeitstempel der Erstellung
        \item \label{experiment_ersteller} Ersteller
        \item Zugehörige \glspl{notiz}
    \end{subanforderung}

    \item \label{notiz_daten}\gls{notiz}daten
    \begin{subanforderung}{D}
        \item \gls{notiz}text
        \item Der \gls{notiz} zugehörige \glspl{tag}
        \item Zeitstempel der Erstellung
        \item \label{notiz_ersteller} Ersteller der \gls{notiz}
        \item Zusätzliche Mediendaten \ref{mediendaten}
    \end{subanforderung}

    \item \label{mediendaten} Mediendaten
    \begin{subanforderung}{D}
        \item Bilddaten
        \item Videodaten
        \item Audiodaten
    \end{subanforderung}

    \item Nutzerdaten
    \begin{subanforderung}{D}
        \item Benutzername
        \item Zugehörigkeit zu Benutzergruppen
        \item Echter Name
        \item Profilbild
        \item Favorisierte \glspl{experiment}
    \end{subanforderung}
\end{anforderung}
\newpage
