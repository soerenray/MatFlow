\documentclass{scrartcl}

\usepackage[utf8]{inputenc} % use utf8 file encoding for TeX sources
\usepackage[T1]{fontenc}    % avoid garbled Unicode text in pdf
\usepackage[german]{babel}  % german hyphenation, quotes, etc
\usepackage{hyperref} % for pdf conversion (links etc.)
\usepackage{csquotes} % provides \enquote{} macro for "quotes"
\usepackage[nonumberlist]{glossaries}  % provides glossary commands

\hypersetup{
    pdftitle={Pflichtenheft},
}
\newcommand{\class}[1]{\begin{center}\Large\bfseries Class: #1 \end{center}}
\newcommand{\method}[1]{\item \textbf{#1}}
\newenvironment{methodenv}[1]{\paragraph{#1}\begin{itemize}}{\end{itemize}}
\newcommand{\smallPara}[1]{\subparagraph{\small #1}}

\newenvironment{dataTable}{\begin{tabular}{|p{4cm}|p{2cm}|p{6cm}|}}{\end{tabular}}

\title{'NAME PRODUKT'
        Workflowanwendung für Machine Learning Experimente}
\author{Florian Küfner, Soeren Raymond, Alessandro Santospirito, \\ 
        Lukas Wilhelm, Nils Wolters} 
\date{November 2021}

\begin{document}

% Titel
\maketitle
\pagebreak
\setcounter{tocdepth}{2}
\tableofcontents
\pagebreak

% Inhalt
\section{Einleitung}

% schreiben wir am Ende

\newpage
\input{zielbestimmung}
\section{Produkteinsatz}
\subsection{Anwendungsbereiche}
    \begin{itemize}
    Die Anwendung soll die Entwicklung und die Ausführung von 'machine learning'-Modellen und Material Simulationen automatisieren. Dazu werden repetetive Abläufe, die zur Ausführung der Simulationen oder zum Trainieren der Modelle benötigt werden, als Workflow erstellt und dann selbständig vom Server ausgeführt. Die Erstellung der Workflows erfolgt mithilfe einer Weboberfläche und setzt keine Programmierkentnisse voraus. Eine Verwaltung der Benutzeraccounts, Festlegung der Hardware-Resourcen und detaillierte Informationen zu den einzelnen Prozessen ist ebenfalls gegeben.
    \end{itemize}
\subsection{Zielgruppe}
    \begin{itemize}
    Das Programm soll von den Mitarbeitern und Studenten im Institut für Materialwissenschaften verwendet werden. Es basiert auf den am KIT benutzten Simulationsprogrammen und den dazu benötigten Dateiformaten. 
    \end{itemize}
\subsection{Betriebsbedingungen}
    \begin{itemize}
    Es werden die vom KIT bereitgestellten Softwareprogramme sowie Hardware-Strukturen benötigt.
    \end{itemize}
\newpage
\section{Produktumgebung}
\subsection{Software}
    \begin{itemize}
    \end{itemize}
\subsection{Hardware}
    \begin{itemize}
    \end{itemize}
\subsection{Schnittstellen}
    \begin{itemize}
    \end{itemize}
\newpage
%Soeren TODO
\section{Produktfunktionen}
    \begin{itemize}
        \item \cleveref{/M10/} wird umgesetzt durch:
        \begin{itemize}
            \item FA10 Machine Learning Workflow erstellen
            \item FA20 Workflow bearbeiten
        \end{itemize}
        \item \cleveref{/M20/} wird umgesetzt durch:
        \begin{itemize}
            \item FA30 Der Benutzer kann die Parameter(Hardwareressourcen, Hyperparameter, ..) für den Workflow konfigurieren
            \item F430 Parameter auf dem Server speichern
        \end{itemize}
        \item \cleveref{/M30/} wird umgesetzt durch:
        \begin{itemize}
            \item FA50 Ein Workflow von dem Server importieren
            \item FA60 Workflow auf dem Server speichern
        \end{itemize}
        \item \cleveref{/M40/} wird umgesetzt durch:
        \begin{itemize}
            \item FA70 Template erstellen
            \item FA80 Template auf dem Server speichern
            \item FA90 Template von dem server importieren
        \end{itemize}
        \item \cleveref{/M50/} wird umgesetzt durch:
        \begin{itemize}
            \item F100 visuelle Darstellung eines Workflows
            \item F110 detailreichere Darstellung eines Workflowitems
        \end{itemize}
        \item \cleveref{/M60/} wird umgesetzt durch:
        \begin{itemize}
            \item F120 Benutzen der Software über Weboberfläche
        \end{itemize}
        \item \cleveref{/M70/} wird umgesetzt durch:
        \begin{itemize}
            \item F130 Eingeben der persönlichen Login-Daten
            \item F140 Validieren der Login-Daten
        \end{itemize}
        \item \cleveref{/M80/} wird umgesetzt durch:
        \begin{itemize}
            \item F150 Erstellen eines neuen Accounts
            \item F160 Festlegen eines Passworts
        \end{itemize}
        \item \cleveref{/M90/} wird umgesetzt durch:
        \begin{itemize}
            \item F170 Zurücksetzen des Passworts
        \end{itemize}
        \item \cleveref{/M100/} wird umgesetzt durch:
        \begin{itemize}
            \item F180 Anzeigen aller Benutzeraccounts
            \item F190 Löschen eines Benutzeraccounts
            \item F200 Bearbeiten eines Benutzeraccounts
        \end{itemize}
        \item \cleveref{/M110/} wird umgesetzt durch:
        \begin{itemize}
            \item F210 Anzeigen der Systemressourcen(Prozessor, RAM, ..)
        \end{itemize}
        \item \cleveref{/M120/} wird umgesetzt durch:
        \begin{itemize}
            \item F220 Anzeigen des Status' der Datenbankverbindung
        \end{itemize}
        \item \cleveref{/M130/} wird umgesetzt durch:
        \begin{itemize}
            \item F230 Benachrichtigungsfunktion bei Fehlschlägen des Workflows
        \end{itemize}
        \item \cleveref{/M140/} wird umgesetzt durch:
        \begin{itemize}
            \item F240 Anzeigen aller auf dem Server gespeicherten Workflows
        \end{itemize}
        
    \end{itemize}
\newpage
%\section{Produkt-Daten}
% unter Produktdaten geht es darum, welche Daten gespeichert und genutzt werden,
% aber nicht wie. Der Datenbankaufbau wird Teil des Entwurfsphase sein.


\newcommand{\todo}{\textcolor{red}{\textbf{ TODO }}}
\newcommand{\discuss}{\textcolor{green}{\textbf{ DISCUSS }}}
\newcommand{\intern}[1]{<\emph{#1}>}
\newcommand{\head}[1]{\vspace{0.2cm}\item[\textbf{#1}]\ \vspace{0.1cm}}
\newcommand{\kriterium}[2]{\vspace{0.2cm}\item[\textbf{#1}] \textbf{\emph{#2} wird umgesetzt durch:\ }\vspace{0.1cm}}

\newenvironment{anforderung}[1]{
	\begin{enumerate}[label=/#1\arabic*0/, leftmargin=1.6cm, noitemsep]}
	{\end{enumerate}
}
\newenvironment{test_env}[1]{
	\begin{enumerate}[label=/#1\arabic*0/, leftmargin=1.6cm]}
	{\end{enumerate}
}
\newenvironment{anforderung_start_at}[2]{\begin{enumerate}[label=/#2\arabic*0/, start={#1}, leftmargin=1.6cm, noitemsep] }{\end{enumerate}}
%\newenvironmentx*{subanforderung}[2][1=1]
%    {\begin{enumerate}[label=/#2\arabic{enumi}\arabic*/, start={#1}]}
%    {\end{enumerate}}





\begin{anforderung}{D}
    \item \label{experiment_daten}\gls{experiment}daten
    \begin{subanforderung}{D}
        \item Name
        \item Beschreibung
        \item Parameter
        \item Status
        \item \glspl{tag}s
        \item Zeitstempel der Erstellung
        \item \label{experiment_ersteller} Ersteller
        \item Zugehörige \glspl{notiz}
    \end{subanforderung}

    \item \label{notiz_daten}\gls{notiz}daten
    \begin{subanforderung}{D}
        \item \gls{notiz}text
        \item Der \gls{notiz} zugehörige \glspl{tag}
        \item Zeitstempel der Erstellung
        \item \label{notiz_ersteller} Ersteller der \gls{notiz}
        \item Zusätzliche Mediendaten \ref{mediendaten}
    \end{subanforderung}

    \item \label{mediendaten} Mediendaten
    \begin{subanforderung}{D}
        \item Bilddaten
        \item Videodaten
        \item Audiodaten
    \end{subanforderung}

    \item Nutzerdaten
    \begin{subanforderung}{D}
        \item Benutzername
        \item Zugehörigkeit zu Benutzergruppen
        \item Echter Name
        \item Profilbild
        \item Favorisierte \glspl{experiment}
    \end{subanforderung}
\end{anforderung}
\newpage

%\input{systemmodell}
%\input{produkt_leistungen}
%\input{benutzeroberflaeche}
%\input{qualitaetsbestimmungen}
%\input{testfaelle}
%\input{entwicklungsumgebung}

% Glossar
\clearpage
%\printnoidxglossaries

\end{document}
